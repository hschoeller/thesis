% !TeX root = ../main.tex
% =================================================
% Chapter 1: Introduction
% =================================================

\chapter{Introduction}\label{ch:intro}
\chapteroverview{This chapter introduces the key concepts and terms relevant
to this thesis. It provides a broad review of the relevant literature and
establishes the research questions addressed in the subsequent chapters.
Finally, it outlines the structure of the thesis.}

\section{Atmospheric blocking}\label{sec:intro_blocking}
\paragraph{Concept} The term \enquote{atmospheric blocking} emerged in weather forecast and atmospheric research
in the early 20th century initially to describe persistent surface high-pressure
regions \enquote{blocking} the usual eastward progression of low-pressure cyclones
(\textcite{rexBlockingActionMiddle1950}; cf.~also \textcite{lupoAtmosphericBlockingEvents2021} and references therein).
It has since come to be understood as a large-scale disturbance
extending through the whole troposphere and interacting with the stratosphere.
The term \enquote{blocking} or simply \enquote{block} notably does not refer
to its geometric shape as a somewhat coherent singular entity, especially since
they can come in different shapes and sizes and are usually embedded in the
wavelike structure of the flow.

Though there is still a lack of consensus on the exact definition of a blocking
and a coherent unifying theory of them is still
at large \parencite{woollingsBlockingItsResponse2018}, the concept broadly encompasses
the notions of a persistent, quasi-stationary, large-scale anticyclonic flow pattern
that occurs in the mid- to high latitudes. Its persistence implies lifetimes on the
order of days to weeks, which is long compared to typical synoptic weather systems.
It is quasi-stationary in that its characteristic flow pattern and position almost do not change
over its lifetime measured by the usual magnitudes of change on those timescales.
This is why some theories liken the phenomenon to quasi-stationary (standing) planetary
waves analogous to waves in other physical systems \parencite[e.g.,~][]{tungTheoryStationaryLong1979, nakamuraAtmosphericBlockingTraffic2018}.

Any weather situation considered a blocking usually involves considerable displacement
of relatively warm and low \gls{pv} subtropical air masses poleward, introducing
or reinforcing a ridge in the upper-level flow. In severe cases this ridge might
even detach from the subtropical reservoir judging from closed contours of low
\gls{pv} values on isentropic surfaces or, equivalently, high potential temperature
values on surfaces of constant \gls{pv} \parencite{pellyNewPerspectiveBlocking2003}.

\paragraph{Definitions}
Consistent with the rather diffuse understanding introduced above, there is no
universally accepted formal definition of atmospheric blocking, and different
diagnostics are adopted depending on purpose, scale of analysis and, arguably,
school of thought \parencite{barnesMethodologyComparisonBlocking2012,barriopedroApplicationBlockingDiagnosis2010}.
Blockings are frequently defined either by indices specifically designed for them
or as one of a number of different weather regimes which, in turn, result from
some kind of clustering of instantaneous atmospheric states.

Indices may be differentiated by which meteorological field they use and by whether
they identify a blocking based on absolute or anomaly fields. The choice of
index will importantly also determine the precise shape of the detected flow field
and will identify different parts of the blocking \parencite{pinheiroAtmosphericBlockingIntercomparison2019}.
For instance, one very
common group of blocking indices is based on the geopotential height field and
its gradient or the associated synoptic winds. This rather kinematic
perspective exploits the blocking's characteristic meridional flow field as a hallmark
and is based on work by \textcite{lejenasCharacteristicsNorthernHemisphere1983} and
\textcite{tibaldiOperationalPredictabilityBlocking1990}. More dynamically and
object-oriented indices identify anomalies in the fields of \gls{pv} or potential
temperature \parencite{schwierzPerspicaciousIndicatorsAtmospheric2004}, a lens that
was pioneered by \textcite{pellyNewPerspectiveBlocking2003} though they identified
gradient reversals instead of anomalies. Since potential temperature and \gls{pv}
are materially conserved under adiabatic and (in the case of \gls{pv}) inviscid conditions,
air-masses that are designated as belonging to a blocking may be traced back and
forth to understand the formation and decay of the blocking event.

A complementary perspective conceptualises blocking as a preferred large-scale
circulation regime within a reduced phase space of atmospheric variability.
In this view, blocking emerges as one metastable state among several recurrent
weather regimes, particularly in the North Atlantic–European sector
\parencite{vautardMultipleWeatherRegimes1990,michelangeliWeatherRegimesRecurrence1995}.
Studying the atmosphere according to a small number of characteristic regimes or
clusters is probably as old as synoptic meteorology itself and has been a ubiquitous
tool in atmospheric sciences ever since \parencite{hannachiLowFrequencyNonlinearity2017}.
This regime-based framing naturally connects to dynamical systems interpretations
which will be properly introduced later on. From this perspective, persistence
and transition probabilities rather than local anomalies or gradient reversals
in the Eulerian fields define the phenomenon.

Finally, from a wave-dynamical perspective, blocking can
be interpreted as a nonlinear manifestation of amplified, breaking, or quasi-stationary
Rossby waves \parencite{tungTheoryStationaryLong1979,nakamuraAtmosphericBlockingTraffic2018}.
In this framework, blockings are associated with \glspl{rwp} \parencite{wirthRossbyWavePackets2018}
or local maxima of wave activity fluxes will thus be identified based on characteristic
patterns in the wavenumber rather than the physical domain.

The differing definitions—index-based, regime-based, and wave-dynamical—thus highlight
that blocking may be understood alternately as a local kinematic signature, a metastable
circulation state, or a nonlinear planetary-wave configuration, depending on the theoretical
and diagnostic perspective adopted.


\paragraph{Phenomenology} Climatologically, blocking events are more prominent in the Northern Hemisphere
compared to the Southern Hemisphere due a stronger land-sea contrast and orography \parencite{woollingsBlockingItsResponse2018}.
They are also larger, more intense \parencite{nabizadeh3DStructureNorthern2021}
and more frequent in winter than in summer, with a peak frequency
in the North Atlantic sector of roughly every 5--6 days on average.
Aside from the North Atlantic, the North Pacific and, to a lesser extent, the
Eurasian sector are also preferred regions for blocking occurrence in the Northern Hemisphere. 
Their implied temporal and spatial scales lets them exert a major influence on
the mid-latitude circulation and weather on a continental to planetary scale for
whole seasons even \parencite{jensenDynamicAnalysisRecord2015}.

Where they occur, blockings can exert a considerable impact on societies and
ecosystems alike, a feature that numerous publications have drawn attention to
\parencite[e.g.,~][]{kautzAtmosphericBlockingWeather2021,whanInfluenceAtmosphericBlocking2016,pfahlQuantifyingRelevanceAtmospheric2012}.
The implied persistence can lead to sustained weather conditions of
different nature depending on season and location relative to the blocking high.
Their eponymous blocking and deflecting of air masses can favour anomalously and
extremely hot or cold conditions and stationary cyclones on their flanks leading
to anomalously high moisture transport and associated precipitation 
\parencite[e.g.,~][]{lenggenhagerAtmosphericBlocksModulate2019, mohrMultidisciplinaryAnalysisExceptional2023}.

Apart from their effect on the advection of air masses, the high pressure associated 
with blockings favours subsidence and, by extension, cloud-free conditions and
reduced vertical mixing in their centers. Together with reduced wind speeds also in the horizontal,
this frequently leads to heat waves and droughts in summer as well as cold spells in winter.
\textcite{herrmannAtmosphericBlocksIncrease2025} showed that disproportianately many
severe wildfires in the Northern high latitudes were associated with blocked episodes in summer.
Reduced mixing may also entail episodes of poor air quality, all of which put severe
pressure on societal systems and human health \parencite{webberDynamicalImpactRossby2017,popeImpactSynopticWeather2016}.
Prolonged episodes of decreases in wind energy supply and increased energy demand
caused by blockings in winter are considered a major risk for the energy sector,
especially with the transition to renewable sources \parencite{oteroCharacterizingRenewableEnergy2022, vanderwielInfluenceWeatherRegimes2019},
potentially aggravating poor air quality through increased fossil fuel use.  

\paragraph{Interactions} The lack of a unified theory of blockings
is probably caused by their interaction
with processes and earth system components on a wide range of spatial
and temporal scales.

The large-scale nature of blockings is evident from their interrelation with planetary-scale
teleconnections. Given their spatial proximity, a modulating relation between 
blockings and the \gls{nao} \parencite{grayElevenyearSolarCycle2016} and \gls{pna}
\parencite{croci-maspoliAtmosphericBlockingSpacetime2007} for Northern Hemisphere blockings
and the \gls{sam} \parencite{oliveiraNewClimatologySouthern2014} for the Southern
Hemisphere seem obvious. More surprisingly perhaps, tropical teleconnections such as
the \gls{enso} \parencite{zhangENSOrelatedSouthernHemisphere2023, mckennaImpactsNinoDiversity2023, barriopedroRelationshipENSOStratospheric2014},
or the \gls{mjo} \parencite{hendersonInfluenceMaddenJulian2016} have also been linked to
blocking frequency and persistence, highlighting the role of planetary-scale disturbances.

The importance of each of the above mechanisms and teleconnections differ regionally and seasonally,
implying a complex two-way interaction between blockings and other components of the earth system
\parencite{burkhardtPlanetarySynopticscaleInteractions2005, nakamuraRoleHighLowFrequency1997}.
More specifically, blocking dynamics are influenced by the polar cryosphere
through heat and moisture fluxes and accompanying sea-ice anomalies on seasonal scales
\parencite{tyrlisUralBlockingDriving2019,woollingsArcticWarmingAtmospheric2014, drijfhoutSpontaneousAbruptClimate2013}.
There, a strong two-way interaction with polar vortex dynamics and the stratosphere
more generally has been established, mediated in particular by upward wave
activity flux and associated tropopause displacements
\parencite{woollingsAssociationsStratosphericVariability2010, colucciDiagnosticComparisonTropospheric2015, yessimbetPathwaysInfluenceNorthern2022}.
From 27 objectively identified \glspl{ssw} in the Northern Hemisphere
between 1957 and 2001, \textcite{martiusBlockingPrecursorsStratospheric2009}
found that 25 were preceded by blocking events in the North Atlantic sector.

Blockings have also been shown to exert a first order influence on the subpolar 
ocean \parencite{hakkinenAtmosphericBlockingAtlantic2011}. On the other hand, \gls{sst}
patterns and according gradients, at times, provide the necessary baroclinic
forcings and heat and moisture content for blockings to form, controlling their
occurrence frequency and intensity \parencite{christSeaSkyUnderstanding2025, mathewsGulfStreamMoisture2024}.
\textcite{narinesinghUniformSSTWarming2024} however showed that a uniform increase
in \glspl{sst} is responsible for the observed change in blocking frequency in a
climate model rather than a modulation of \gls{sst} gradients. This highlights
the uncertainty associated with the precise preconditioning of \glspl{sst} for blockings
with their according fluxes. In a similar manner, blocking dynamics are also shaped by surface level diabatic
sensible and latent heat fluxes over land \parencite{tillyCalculatedHeightTendencies2008}
and their climatological occurrence is modulated by land-sea contrasts and orography
through initiated planetary waves \parencite{heImpactLandSea2014}.

\paragraph{Predictability}
Atmospheric blocking events are frequently discussed in the context of operational
predictability on subseasonal to seasonal timescales in particular.
On the one hand, this is due to their extraordinary impact on
natural and societal systems outlined above. On the other hand, \gls{nwp} models
have historically struggled to accurately forecast blocking events, especially
their onset and decay \parencite{dandreaNorthernHemisphereAtmospheric1998, matsuedaBlockingPredictabilityOperational2009,tibaldiOperationalPredictabilityBlocking1990}
whereas medium-range foreast skill is sometimes enhanced during blocking episodes
(\cite{spaethFlowDependenceEnsembleSpread2024}; cf.~also \cite{prestel-kupfererPredictabilityMidlatitudeRossbywave2024}
for a similar behaviour for \glspl{rwp}). Based on the weather regime classification
introduced by \textcite{gramsBalancingEuropesWindpower2017} (cf.~\cite{gramsLifeCycleDefinition2026}
for an updated version),
\textcite{buelerYearroundSubseasonalForecast2021} and \textcite{osmanMultimodelAssessmentSubseasonal2023}
provided evidence that blockings are both the best and worst predicted regimes in the Euro-Atlantic sector,
depending on their location with a blocking over Greenland achieving the longest and
blockings over Central Europe or Scandinavia the shortest forecast skill horizon.
In a similar manner, but using a different set of regimes and only data in winter,
\textcite{ferrantiFlowdependentVerificationECMWF2015} found that operational
forecasts struggled most with transitions to a blocking regime over Northern Europe
while also underestimating their persistence.

The issues faced by \gls{nwp} models in forecasting blockings are reflected in
atmospheric models' difficulties in representing them realistically in the first place,
which is apparent from considerable biases in blocking frequency, intensity and persistence
\parencite{daviniNorthernHemisphereAtmospheric2016}. \textcite{leeIncreasingFrequencyPersistence2026}
recently reported that a seasonal large-ensemble prediction model was able to
reproduce Greenland blocking frequency and persistence but failed to reproduce its trend
due to climate change, making the picture even more complex.
It seems like there are multiple causes for the misrepresentation of blockings,
ranging from insufficient model resolution, coupling to other earth systems
and representation of sub-grid scale processes---all of which exacerbated in the case
of climate models \parencite{woollingsBlockingItsResponse2018}. In an evaluation of the
operational seasonal prediction model of the \gls{ecmwf}, \textcite{daviniRepresentationWinterNorthern2021}
found that while improvements have taken place, significant biases remain in the
European sector in particular. They note that resolution increases improved blocking
representation considerably and highlight that errors in the atmospheric system are
most relevant as opposed to errors in ocean representation, though that behaviour seems
to be different in the climate setting.

In an evaluation of \gls{cmip} 6, the latest generation
of climate models, \textcite{maddisonMissingIncreaseSummer2024} found that the
increase in summer blocking over Greenland observed in since the beginning of
the century is not reprocuced and suggest this might be linked to \glspl{sst},
sea ice biases or anthopogenic aerosols. Indeed, given what has been mentioned about
the interaction of atmospheric blockings
with other earth system components varying on longer timescales, it seems natural to expect
that the the predictive skill of models predicting blockings depends on whether they
can accurately represent those components and their interaction with blockings.

\section{Processes and Theories of Atmospheric Blocking}
Given all that has been stated above about the complex phenomenon of atmospheric blockings
from practical perspectives, it is not surprising that there is not a single mathematical
theory explaining all of its facets. That said, a number of mathematical theories have
been proposed that explain different aspects of the phenomenon, based on different
assumptions and perspectives.

\paragraph{Barotropic theories} More traditional theories on blockings focus on barotropic large scale
dynamics---a stance that is justified by the fact that the vertical structure
of atmopsheric blockings is dominantly equivalent-barotropic, with little vertical
variation in temperature and wind \parencite{nabizadeh3DStructureNorthern2021}.
The earliest synoptic work took a primarly diagnostic stance with some likening the phenomenon
to hydrolic jumps in other jet like flows \parencite{rexBlockingActionMiddle1950a}
inspired by the observationally guided understanding of a blockings as a breakdown
of the usual zonal flow \parencite{berggrenAerologicalStudyZonal1949}. These merely
descriptive works set the stage for more quantative theories with some degree of
explanatory power.

In a barotropic flow with external (usually topographic and radiative) forcing
blocking can appear as a stationary (Rossby-) wave phenomenon \parencite{rossbyRelationVariationsIntensity1939}.
This is because it can be shown that for certain Rossby wave constellations, the
waves' group velocity exactly opposes the advection with the mean flow.
Assuming linearized barotropic
dynamics around some climatological flow, this perspective can explain
regions of climatalogical blocking occurrence, but lacks mechanisms necessary to
understand onset and persistence in a way comparable with observations \parencite{hoskinsSteadyLinearResponse1981}.

This was also pointed out by \textcite{charneyMultipleFlowEquilibria1979}, who
proposed a model in which non-linearities were essential for blocking occurrence.
Similar to the notion of stationary waves, their model admitted multiple equilibria
and opened the understanding of blocking as a (quasi-)stationary regime governing
the atmospheric system's dynamics in its vicinity in state space. The model will
be introduced in more detail later. Suffice to say that it offers a mechanism for
regime persistence and sudden transitions, but its applicability is questioned by
its sensitivity to parameter choices \parencite{crommelinMechanismAtmosphericRegime2004}
and whether the real atmosphere possesses regimes in the dynamical sense is debated
\parencite{stephensonExistenceMultipleClimate2004}.

The quasi-stationary wave perspective was advanced by \textcite{tungTheoryStationaryLong1979}
who conceived blockings as waves resonant with topography and land-sea contrast, but
kept in place by specific conditions of the background flow. Non-linearities are
assumed in this model and baroclinic extensions are discussed, though not necessary and
time-dependent amplification explains realistic onset and persistence mechanisms
lacking in the basic Rossby wave concept above.

A separate set of theories argue that due to their nearly equivalent-barotropic
structure, blockings resemple individual vortices interacting with other vortices
in a background flow. \textcite{mcwilliamsApplicationEquivalentModons1980} showed that
the (non-linear) equivalent-barotorpic vorticity equation admits vortex pairs called \enquote{modons}
as solutions similar in structure and behaviour to blockings. Similar results
can be obtained when considering point vortices as abstractions \parencite{mullerApplicationsPointVortex2015}.
In a related view \textcite{hainesEddyForcedCoherentStructures1987}, interpret
synoptic eddies as providing a dipole \gls{pv} forcing that supports the coherent
modon structure against dissipation. The fact that transient eddies can systematically
transfer vorticity to foster large scale flow configurations is well established
\parencite{kuoVorticityTransferRelated1951}, but albeit there is some evidence that such
interactions are relevant in the real world \parencite{yamazakiVortexVortexInteractions2013},
modon theory can at best seen as a paradigm-level illustration, especially considering
that cyclones are scarcely barotropic.

\paragraph{Baroclinic theories}
Blockings appear not in a single layer flow, but in a stratified, vertically sheared
atmosphere in which baroclinic instability continuously generates synoptic-scale
eddies shaping the general circulation. Classical linear instability theory
\parencite[e.g.~][]{charneyDynamicsLongWaves1947, eadyLongWavesCyclone1949} and
energy cycle considerations \parencite{lorenzAvailablePotentialEnergy1955},
demonstrate that in such a setting, generic instabilites convert available
potential energy into eddy kinetic energy on synoptic scales.
Baroclinic theories tackle the central question how these synoptic
eddies feed back onto the planetary-scale flow to produce persistent large-scale
anomalies. In this sense, the baroclinic perspective complements earlier barotropic
frameworks introduced above by seeking a physically consistent explanation for transient
eddies capable of sustaining or destabilizing large-scale flow regimes, and they
provide models with more realistic assumptions than idealized approaches
based on modons or point-vortex dynamics.

The feedbacks between eddies of any scale and the large scale flow is closely tied
to the nature of turbulent cascades in geophysical flows. In fully three-dimensional
turbulence, energy cascades downscale toward dissipation \parencite{popeTurbulentFlows2000}.
In contrast, two-dimensional turbulence may exhibit inverse energy cascades,
transferring kinetic energy to larger scales \parencite{boffettaTwoDimensionalTurbulence2012}.
Quasi-geostrophic turbulence appropriate for a strongly stratified, rapidly rotating
atmosphere shares aspects of both: baroclinic instability injects energy at synoptic
scales, while nonlinear interactions transfer kinetic energy upscale and potential
enstrophy downscale \parencite{kittungDifferences2DQG2003}.
This upscale energy transfer encapsulates the existence of feedback mechanisms
described above in an abstract way (in spectra space rather than physical space).
In this view, baroclinic theories of blocking thus interpret persistence not as
a steady equilibrium, but as the large-scale manifestation of sustained, yet
principally intermittent, eddy fluxes and inverse energy transfer within a
stratified turbulent flow.


Being an elementary feature of the large-scale atmospheric circulation,
it comes as no surprise that blocking phenomena have been shown to appear even in
simplified systems based on barotropic assumptions \parencite[e.g.,~][]{austinBlockingMiddleLatitude1980, legrasPersistentAnomaliesBlocking1985,charneyMultipleFlowEquilibria1979}.
There, blockings arise from interactions between Rossby waves with themselves or
with the mean flow. Some theories highlight interactions between individual
vortices \parencite{yamazakiVortexVortexInteractions2013, mullerApplicationsPointVortex2015, mcwilliamsApplicationEquivalentModons1980},
digressing from the usual wave-centered view of mid-latitude dynamics.

In a dry, barotropic framework, a blocking's quasi-stationarity and persistence can be explained by
a (low-frequency) balance between wave propagation and the advection by the mean flow
\parencite{mullenLocalBalancesVorticity1986} and a (high-frequency) interaction
with transient eddies \parencite{cashDynamicalProcessesBlock2000,nakamuraRoleHighLowFrequency1997},
though transient eddies in the real atmosphere are scarcely dry and barotropic.
Nevertheless, consistent with the barotropic view above,
the decay of blockings is then associated with the disruption
of this balance through advected disturbances or gradual diffusive erosion of the
meridonal \gls{pv} gradient.

Baroclinic theories emphasize the role of the vertical structure of the flow and
have frequently highlighted the importance of upstream disturbances and wave-interactions
for blocking onset and decay. In a simple two-layer model, \textcite{luoPlanetaryscaleBaroclinicEnvelope2000}
observed that an upstream baroclinic eddy can bring a high-wavenumber Rossby wave
into phaselock. In this view, baroclinicity is important to bring about the necessary
amplification of a wave disturbance to reach a blocking configuration, and eventually
dissipate it again.

The evident interrelation of blockings with the ocean, cryosphere, land and
convective processes already alludes to the fact that moist processes are just as important
as dry dynamics in understanding blocking dynamics \parencite{dolores-tesillosRoleMoistDry2025}.
This aspect was historically
overlooked mainly due to difficulties in representing moist processes in models
and precipitation's small-scale nature compared to synoptic scales, but also since
blocking's characteristic large-scale subsidence typically suppresses convection.

\section{The atmosphere as a dynamical system}

in which blocking corresponds to a quasi-stationary attractor or a long-lived excursion
within a weakly unstable manifold of the large-scale flow \parencite{charneyMultipleFlowEquilibria1979,legrasPersistentAnomaliesBlocking1985}.

\section{The Charney-deVore model}

\section{Data assimilation}

\section{Warm conveyor belts}\label{sec:intro_wcb}

\paragraph{Concept} \Glspl{wcb} are coherent, ascending airstreams that transport warm, moist air
from the warm sector of an extratropical cyclone in the lower troposphere
to the upper troposphere and lower stratosphere \parencite{schwenkRoleAscentTimescales2024}.
Its principle existence and effects
where present already in the classical theory of extratropical cyclones at the
polar front by \textcite{BjerknesSolberg1922}, but conceptualisation and dedicated
investigation took place later within the framework of the three-dimensional cyclone
conveyor-belt model \parencite[e.g.,~][]{browningStructureRainbandsMidlatitude1973}.
They were objects of interest originally due to their associated rain bands, but have
since also been recognized as important dynamical drivers of synoptic scale baroclinic dynamics
\parencite{madonnaWarmConveyorBelts2014a, binderRoleWarmConveyor2016} and, in effect,
the large-scale mid-latitude flow \parencite{vishnupriyaInteractionWarmConveyor2025, blanchardMidlevelConvectionWarm2021}.
Due to their considerable forcing and moisture content, \glspl{wcb} commonly account
for precipitation extremes in the extratropics on
daily timescales \parencite{pfahlWarmConveyorBelts2014}.

\Glspl{wcb} are typically identified in a Lagrangian sense as traced air parcels that ascend
rapidly while undergoing substantial latent heating. A common criterion to select
\glspl{wcb} is to require an ascent of at least 600 hPa within 2 days \parencite[e.g.,~][]{madonnaWarmConveyorBelts2014a, }
which goes back to seminal work by \textcite{wernliLagrangianbasedAnalysisExtratropical1997} and
\textcite{wernliLagrangianbasedAnalysisExtratropical1997a}. Identifying \glspl{wcb}
from trajectories is, of course, only possible post-hoc, which presents a problem
since \glspl{wcb} are also of practical interest in weather forecasting. This is
why \textcite{quintingEuLerianIdentificationAscending2022a} have recently developed
a machine-learning based approach to identify \glspl{wcb} in an Eulerian framework.
Given relevant meteorological fields, their method can identify \glspl{wcb} in their
inflow, ascent and outflow phases with good accuracy, especially in regions of high
climatological \gls{wcb} occurrence.

\paragraph{Dynamics and relevance} \Glspl{wcb} are dynamically important since they represent a major pathway
During ascent, latent heat release strongly modifies the thermodynamic and
dynamical structure of the flow, leading to changes in potential vorticity and
the amplification or redistribution of upper-level wave features.
In particular, diabatically generated low-level potential vorticity anomalies
and reduced upper-level potential vorticity can reinforce downstream ridges
and thereby influence the evolution of large-scale flow patterns.

Because they link boundary-layer moisture reservoirs with the upper-tropospheric
waveguide, WCBs play a central role in mediating interactions between synoptic-scale
baroclinic development and the planetary-scale circulation.
Beyond their dynamical significance, they are closely associated with elongated
cloud bands and heavy precipitation, making them key agents of both moisture
transport and large-scale flow modification in the mid-latitudes.

The prominence of \glspl{wcb} especially for blocks over the oceans leaves an imprint
on their climatology where a significant ascending motion it visible on the western
flank of the blocking high compared to subsidence in its center and eastern flank
\parencite{nabizadeh3DStructureNorthern2021}.


\section{Predictability and uncertainty}\label{sec:intro_predictability}

\subsection{Predictability in dynamical systems}\label{sec:intro_predict_dyn}
% Error growth, flow dependence, trajectory vs statistical predictability (short).

\subsection{Sources of uncertainty}\label{sec:intro_unc_sources}
% Initial conditions vs model error; forecast vs analysis uncertainty (short).
% (Optional: table taxonomy of uncertainty sources.)

\subsection{Stochastic parameterizations}\label{sec:intro_stoch}
% Why stochastic physics exists; documented circulation/regime impacts (high-level).

\subsection{Assimilation uncertainty}\label{sec:intro_assim_unc}

\section{Aims and thesis roadmap}\label{sec:intro_roadmap}

\subsection{Research questions}\label{sec:intro_questions}
% 2--4 bullet points.

\subsection{Contributions}\label{sec:intro_contrib}
% 3--6 sentences mapping to Paper I/II/III (short).

\subsection{Thesis structure}\label{sec:intro_structure}
% Chapter-by-chapter overview; use \cref for references to chapters.

% Example (fill in your actual chapter labels):
% \Cref{ch:paper1} ... \Cref{ch:paper2} ... \Cref{ch:paper3} ... \Cref{ch:conclusion} ...
