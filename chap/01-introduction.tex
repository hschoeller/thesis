% !TeX root = ../main.tex
% =================================================
% Chapter 1: Introduction
% =================================================

\chapter{Introduction} % LaTeX chapter title
\label{chap:intro}

\section{Atmospheric Blocking}
\label{sec:blocking}
Atmospheric blocking represents a critical phenomenon in the dynamics of Earth's atmosphere,
characterized by the temporary obstruction of prevailing westerlies in the mid-latitudes,
potentially influencing weather on a planetary scale \parencite{lupoAtmosphericBlockingEvents2021}.
These events are notable for their role in causing extreme weather conditions,
such as heatwaves, cold spells, and sustained periods of precipitation,
impacting both human activities and natural ecosystems
\parencite{kautzAtmosphericBlockingWeather2022, pfahlQuantifyingRelevanceAtmospheric2012}.
The mechanisms leading to atmospheric blocking are complex,
involving interactions between the atmosphere,
cryosphere \parencite{tyrlisUralBlockingDriving2019, woollingsArcticWarmingAtmospheric2014},
ocean \parencite{drijfhoutSpontaneousAbruptClimate2013, hakkinenAtmosphericBlockingAtlantic2011},
and land \parencite{heImpactLandSea2014, kurganskyAtmosphericCirculationResponse2020, tillyCalculatedHeightTendencies2008},
and are a subject of ongoing research.
Despite their significance,
predicting atmospheric blocking events and their impacts remains a challenge,
due to the inherent variability and the multifaceted processes that govern their formation,
maintenance, and dissipation.
Weather and climate models' inability to correctly represent blocking
\parencite{matsuedaBlockingPredictabilityOperational2009}
also causes considerable uncertainties in its predicted response to climate change
\parencite{woollingsBlockingItsResponse2018}.

\subsection{Blockings and Warm Conveyor Belts}
\label{sec:blockWCB}

Though traditionally conceived as an adiabatic phenomenon,
recent studies have pointed to the role of diabatic processes in blockings
\parencite{croci-maspoliKeyDynamicalFeatures2009, hauserHolisticUnderstandingBlocked2023, pfahlImportanceLatentHeat2015, tillyCalculatedHeightTendencies2008}.
More specifically,
a line of study has been concerned with moist processes,
arguing that a considerable fraction of the air masses constituting the blocking
originate from warm sectors of neighboring surface cyclones
and travel into the blocking via \glspl{wcb}
\parencite{pfahlImportanceLatentHeat2015, steinfeldRoleLatentHeating2019, yamamotoOceanicMoistureSources2021, zschenderleinLagrangianAnalysisUppertropospheric2020}.
\glspl{wcb} are moist, rapidly ascending air streams
and are subject to research in their own right
\parencite[e.g.,~][]{joosWarmConveyorBelts2023, madonnaWarmConveyorBelts2014a}
i.a.~due to their contribution to forecast uncertainty
\parencite{madonnaVerificationNorthAtlantic2015, picklWarmConveyorBelts2023, wandelSystematicEvaluationWarm2021}.
According to \textcite{steinfeldRoleLatentHeating2019},
the air parcels conveyed by such \glspl{wcb} considerably influence the blocking
by stabilizing and potentially intensifying the negative \gls{pv} anomaly
characteristic for the blocking,
especially in the onset and maintenance stages of its lifecycle
(negative \gls{pv} anomalies are associated with anticyclonal circulation and thus high pressure).
This happens, firstly,
through a material change in \gls{pv} of the respective air masses
via latent heating induced, cross-isentropic vertical motion
and, secondly,
through the divergent outflow of the guiding air streams in the upper troposphere.
Evidence that moist processes might become more important
in a warmer and moister climate underlines the need for further investigation
\parencite{steinfeldResponseMoistDry2022}.

The notion of a \gls{wcb} suggests a geometrically coherent flow of the air masses involved,
but such a coherent behavior has not been explicitly addressed in the studies above.
The backward trajectories attributed to such air streams
are typically identified using thresholds for ascent (change in pressure)
or diabatic heating (change in potential temperature),
such that a common pathway is not guaranteed.
In addition,
the methodology employed does not allow for statements about the size,
location,
time of existence
or even the number of \glspl{wcb} implicated in the process.
Another drawback of the identification method is its subjective nature,
implied by selecting an arbitrary threshold for the decrease in pressure
(\cite{madonnaWarmConveyorBelts2014a} use 600 hPa in 48 h)
or increase in potential temperature the air parcels have experienced
(usually a threshold of $\Delta \theta > 2$ K is used).
To the best of our knowledge,
an objective identification method for coherent medium-to-large scale air streams
that allows addressing these drawbacks
has not been presented in the context of atmospheric sciences.

In the study presented in \cref{chap:coherence},
we want to tackle these issues focusing on the exact spatio-temporal nature
of trajectories passing through a blocking
and how individual trajectories align with each other.
By grouping the set of trajectories into subsets (clusters)
based on only their geometric coherence,
we identify synoptic scale air streams
and analyze their individual dynamical properties
and interrelation with the large scale flow configuration.
The perspective introduced also allows for statements
about the overall spatial configuration and coherence of the air parcels traced,
especially with respect to the blockings' lifecycles.
A blocking's stabilizing and dispersion-suppressing nature
affects individual air parcels,
which can be identified through our Lagrangian lens
\parencite{ehstandCharacteristicSignaturesNorthern2021}.

To accomplish the above,
we make use of the mathematical concept of coherent sets.
A coherent set is a time-evolving region in the state space of a dynamical system
that keeps its geometric integrity to a large degree.
This is particularly interesting when the underlying system is such
that individual trajectories diverge
and any set eventually becomes highly filamented under evolution through the dynamics.
We characterize coherent sets based on their stability under small perturbations.
Coherent sets hence resist dispersion,
persist longer in complicated flows
and have thus an outstanding effect on the transport of physically relevant quantities.
The fundamental idea is that in a coherent set
the individual trajectories stay relatively close together as a whole,
such that particles remain within the coherent set
when they are subjected to small random perturbations.
In contrast,
if a time-evolving set is filamented,
particles are likely to leave the set under the influence of random noise
-- and hence mix with its exterior.
To formalize this heuristics and numerically compute coherent sets,
we follow the work of \textcite{banischUnderstandingGeometryTransport2017}
(based on theory developed in~\textcite{froylandAnalyticFrameworkIdentifying2013,froylandDynamicIsoperimetryGeometry2015}),
which characterizes the robustness of coherent sets under small perturbations
using a data-analysis framework called diffusion maps.
Their method yields a single time-averaged operator
whose spectral properties contain the necessary information
to extract coherent sets.

The visual idea of coherence (i.e., ``trajectories staying together'')
can be cast into diverse objectives,
which then can be used to partition available trajectory data into such sets.
For a sample of the ideas that have been implemented
please consult~\textcite{froylandRoughandreadyClusterbasedApproach2015, allshouseLagrangianBasedMethods2015, hadjighasemSpectralclusteringApproachLagrangian2016, schlueter-kuckCoherentStructureColouring2017, padberg-gehleNetworkbasedStudyLagrangian2017, koltaiLargeDeviationsSemidistances2018}.
A related notion is that of Lagrangian coherent structures
\parencite{hallerCoherentLagrangianVortices2013,hallerLagrangianCoherentStructures2015},
aiming to describe barriers of transport.
For their computation from finite trajectory data,
see for instance~\textcite{mowlaviDetectingLagrangianCoherent2022}.

The main foci of this study are the adaptation of the methodological framework
of \textcite{banischUnderstandingGeometryTransport2017}
to real world trajectory data of air parcels in the atmosphere
and its deployment for case studies of atmospheric blockings.
This is motivated by the question of spatial coherence in \glspl{wcb}
in the context of blocking.
As such,
the presented work does not endeavour to give a one-size-fits-all solution
to the problem of finding coherent air streams
regardless of scale, geographic location and synoptic condition,
but should rather be understood as a proof of concept.
We think atmospheric blocking is a phenomenon well suited
for the application of the developed methodology,
since it is large enough to feature a handful of distinct, coherent air streams
(given the resolution of our data),
but small enough to be well resolved by the number of trajectories computationally feasible.
We have thoroughly analyzed two cases of atmospheric blocking
differing with respect to both geographical location and time of year
and show that differences between the two examples
at similar points in their lifecycles
are notably smaller than differences within the same example
between different points in the lifecycles.
This goes for the occurrence of \glspl{wcb},
the overall trajectory density
and the traced air masses' filamentation.

\subsection{Blockings in Models}


\section{Uncertainty}
\subsection{Mathematics of Uncertainty}
\subsection{Uncertainty and Error Growth}