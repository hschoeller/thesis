% !TeX root = ../main.tex
% =================================================
% Chapter 1: Introduction
% =================================================

\chapter{Introduction}\label{ch:intro}
\chapteroverview{This chapter introduces the key concepts and terms relevant
to this thesis. It provides a broad review of the relevant literature and
establishes the research questions addressed in the subsequent chapters.
Finally, it outlines the structure of the thesis.}

\section{Atmospheric blocking}\label{sec:intro_blocking}
\paragraph{Concept} The term \enquote{atmospheric blocking} emerged in weather forecast and atmospheric research
in the early 20th century initially to describe persistent surface high-pressure
regions \enquote{blocking} the usual eastward progression of low-pressure cyclones
(\textcite{rexBlockingActionMiddle1950}; also \textcite{lupoAtmosphericBlockingEvents2021} and references therein).
It has since come to be understood as a large-scale anticyclonic flow pattern
extending through the whole troposphere and interacting with the stratosphere \parencite{yessimbetPathwaysInfluenceNorthern2022,woollingsAssociationsStratosphericVariability2010,colucciDiagnosticComparisonTropospheric2015}.
The term \enquote{blocking} or simply \enquote{block} notably does not refer
to its geometric shape as a somewhat coherent singular entity, especially since
they can come in different shapes and sizes and are usually embedded in the
wavelike structure of the flow.

Though there is still a lack of consensus on the exact definition of a blocking
and a coherent unifying theory of them is still
at large \parencite{woollingsBlockingItsResponse2018}, the concept broadly encompasses
the notions of a persistent, quasi-stationary, large-scale anticyclonic flow pattern
that occurs in the mid- to high latitudes. Its persistence implies lifetimes on the
order of days to weeks, which is long compared to typical synoptic weather systems.
It is quasi-stationary in that its characteristic flow pattern and position almost do not change
over its lifetime measured by the usual magnitudes of change on those timescales.
This is why some theories liken the phenomenon to quasi-stationary (standing) planetary
waves analogous to waves in other physical systems \parencite[e.g.,~][]{tungTheoryStationaryLong1979, nakamuraAtmosphericBlockingTraffic2018}.

Any weather situation considered a blocking usually involves considerable displacement
of relatively warm and low \gls{pv} subtropical air masses poleward, introducing
or reinforcing a ridge in the upper-level flow. In severe cases this ridge might
even detach from the subtropical reservoir judging from closed contours of low
\gls{pv} values on isentropic surfaces or, equivalently, high potential temperature
values on surfaces of constant \gls{pv} \parencite{pellyNewPerspectiveBlocking2003}.

\paragraph{Definitions}
Consistent with the rather diffuse understanding introduced above, there is no
universally accepted formal definition of atmospheric blocking, and different
diagnostics are adopted depending on purpose, scale of analysis and, arguably,
school of thought \parencite{barnesMethodologyComparisonBlocking2012,barriopedroApplicationBlockingDiagnosis2010}.
Blockings are frequently defined either by indices specifically designed for them
or as one of a number of different weather regimes which, in turn, result from
some kind of clustering of instantaneous atmospheric states.

Indices may be differentiated by which meteorological field they use and by whether
they identify a blocking based on absolute or anomaly fields. The choice of
index will importantly also determine the precise shape of the detected flow field
and will identify different parts of the blocking \parencite{pinheiroAtmosphericBlockingIntercomparison2019}.
For instance, one very
common group of blocking indices is based on the geopotential height field and
its gradient or the associated synoptic winds. This rather kinematic
perspective exploits the blocking's characteristic meridional flow field as a hallmark
and is based on work by \textcite{lejenasCharacteristicsNorthernHemisphere1983} and
\textcite{tibaldiOperationalPredictabilityBlocking1990}. More dynamically and
object-oriented indices identify anomalies in the fields of \gls{pv} or potential
temperature \parencite{schwierzPerspicaciousIndicatorsAtmospheric2004}, a lens that
was pioneered by \textcite{pellyNewPerspectiveBlocking2003} though they identified
gradient reversals instead of anomalies. Since potential temperature and \gls{pv}
are materially conserved under adiabatic and (in the case of \gls{pv}) inviscid conditions,
air-masses that are designated as belonging to a blocking may be traced back and
forth to understand the formation and decay of the blocking event.

A complementary perspective conceptualises blocking as a preferred large-scale
circulation regime within a reduced phase space of atmospheric variability.
In this view, blocking emerges as one metastable state among several recurrent
weather regimes, particularly in the North Atlantic–European sector
\parencite{vautardMultipleWeatherRegimes1990,michelangeliWeatherRegimesRecurrence1995}.
Studying the atmosphere according to a small number of characteristic regimes or
clusters is probably as old as synoptic meteorology itself and has been a ubiquitous
tool in atmospheric sciences ever since \parencite{hannachiLowFrequencyNonlinearity2017}.
This regime-based framing naturally connects to dynamical systems interpretations
which will be properly introduced later on. From this perspective, persistence
and transition probabilities rather than local anomalies or gradient reversals
in the Eulerian fields define the phenomenon.

Finally, from a wave-dynamical perspective, blocking can
be interpreted as a nonlinear manifestation of amplified, breaking, or quasi-stationary
Rossby waves \parencite{tungTheoryStationaryLong1979,nakamuraAtmosphericBlockingTraffic2018}.
In this framework, blockings are associated with \glspl{rwp} \parencite{wirthRossbyWavePackets2018}
or local maxima of wave activity fluxes will thus be identified based on characteristic
patterns in the wavenumber rather than the physical domain.

The differing definitions—index-based, regime-based, and wave-dynamical—thus highlight
that blocking may be understood alternately as a local kinematic signature, a metastable
circulation state, or a nonlinear planetary-wave configuration, depending on the theoretical
and diagnostic perspective adopted.


\paragraph{Phenomenology} Climatologically, blocking events are more prominent in the Northern Hemisphere
compared to the Southern Hemisphere due a stronger land-sea contrast and orography \parencite{woollingsBlockingItsResponse2018}.
They are also more frequent in winter than in summer, with a peak frequency
in the North Atlantic sector of roughly every 5--6 days on average.
Aside from the North Atlantic, the North Pacific and, to a lesser extent, the
Eurasian sector are also preferred regions for blocking occurrence in the Northern Hemisphere. 

Where they occur, blockings can exert a considerable impact on societies and
ecosystems alike, a feature that numerous publications have drawn attention to
\parencite[e.g.,~][]{kautzAtmosphericBlockingWeather2021,whanInfluenceAtmosphericBlocking2016,pfahlQuantifyingRelevanceAtmospheric2012}.
The implied persistence can lead to sustained weather conditions of
different nature depending on season and location relative to the blocking high.
Their eponymous blocking and deflecting of air masses can favour anomalously and
extremely hot or cold conditions and stationary cyclones on their flanks leading
to anomalously high moisture transport and associated precipitation 
\parencite[e.g.,~][]{mohrMultidisciplinaryAnalysisExceptional2023}.
Apart from their effect on the advection of air masses, the high pressure associated 
with blockings favours subsidence and, by extension, cloud-free conditions and
reduced vertical mixing in their centers. Together with reduced wind speeds also in the horizontal,
this frequently leads to heat waves and droughts in summer as well as cold spells in winter.
Reduced mixing may also entail episodes of poor air quality, all of which put severe
pressure on societal systems and human health \parencite{webberDynamicalImpactRossby2017,popeImpactSynopticWeather2016}.
Prolonged episodes of decreases in wind energy supply and increased energy demand
caused by blockings in winter are considered a major risk for the energy sector,
especially with the transition to renewable sources \parencite{oteroCharacterizingRenewableEnergy2022, vanderwielInfluenceWeatherRegimes2019},
potentially aggravating poor air quality through increased fossil fuel use.  

\paragraph{Processes and Interactions} The lack of a unified theory of blockings is probably caused by their interaction
with processes and earth system components on a wide range of spatial
and temporal scales. More traditional theories on blockings focus on large scale
dynamics. Being an elementary feature of the large-scale atmospheric circulation,
it comes as no surprise that blocking phenomena have been shown to appear in
simplified systems based on barotropic assumptions \parencite[e.g.,~][]{austinBlockingMiddleLatitude1980, legrasPersistentAnomaliesBlocking1985,charneyMultipleFlowEquilibria1979}.
There, blocking arise from interactions between Rossby waves with themselves or
with the mean flow. Some theories highlight interactions between individual
vortices \parencite{yamazakiVortexVortexInteractions2013, mullerApplicationsPointVortex2015},
digressing from the usual wave-centered view of mid-latitude dynamics.

The large-scale nature of blockings is also evident from their interrelation with planetary-scale
teleconnections. Given their spatial proximity, a modulating relation between 
blockings and the \gls{nao} \parencite{grayElevenyearSolarCycle2016} and \gls{pna}
\parencite{croci-maspoliAtmosphericBlockingSpacetime2007} for Northern Hemisphere blockings
and the \gls{sam} \parencite{oliveiraNewClimatologySouthern2014} for the Southern
Hemisphere seem obvious. More surprisingly perhaps, tropical teleconnections such as
the \gls{enso} \parencite{zhangENSOrelatedSouthernHemisphere2023, mckennaImpactsNinoDiversity2023, barriopedroRelationshipENSOStratospheric2014},
or the \gls{mjo} \parencite{hendersonInfluenceMaddenJulian2016} have also been linked to
blocking frequency and persistence, highlighting the role of planetary-scale disturbances.

The importance each of the above mechanisms and teleconnections differ regionally and seasonally,
implying a complex two-way interaction between blockings and other components of the earth system
\parencite{burkhardtPlanetarySynopticscaleInteractions2005, nakamuraRoleHighLowFrequency1997}.
More specifically, blocking dynamics are influenced by the polar cryosphere
through heat and moisture fluxes and accompanying sea-ice anomalies on seasonal scales
\parencite{tyrlisUralBlockingDriving2019,woollingsArcticWarmingAtmospheric2014, drijfhoutSpontaneousAbruptClimate2013}.
There, a strong two-way interaction with polar vortex dynamics and the stratosphere
more generally has been established, mediated in particular by upward wave
activity flux and associated tropopause displacements
\parencite{colucciDiagnosticComparisonTropospheric2015, yessimbetPathwaysInfluenceNorthern2022}.
From 27 objectively identified \glspl{ssw} between 1957 and 2001, \textcite{martiusBlockingPrecursorsStratospheric2009}
found that 25 were preceded by blocking events in the North Atlantic sector.
Blockings have also been shown to exert a first order influence on the subpolar 
ocean \parencite{hakkinenAtmosphericBlockingAtlantic2011}. On the other hand, \gls{sst}
patterns and according gradients, at times, provide the necessary baroclinic
forcings and heat and moisture content for blockings to form, controlling their
occurrence frequency and intensity \cite{christSeaSkyUnderstanding2025, mathewsGulfStreamMoisture2024}.
In a similar manner, blocking dynamics are also shaped by surface level diabatic
sensible and latent heat fluxes over land \parencite{tillyCalculatedHeightTendencies2008}
and their climatological occurrence modulated by land-sea contrasts and orography
through initiated planetary waves \parencite{heImpactLandSea2014}.
The above already alludes to the fact that moist processes are just as important
as dry dynamics in understanding blocking dynamics.

\paragraph{Predictability}

\subsection{Impacts and relevance}\label{sec:intro_blocking_impacts}
% Extremes, persistence, S2S relevance, societal impacts.

\subsection{Diagnostics and perspectives}\label{sec:intro_blocking_diag}
% Blocking indices (brief), Eulerian vs regime-based framing, why persistence matters.
% (Optional: short table comparing diagnostics; optional schematic figure.)

\section{Predictability and uncertainty}\label{sec:intro_predictability}

\subsection{Predictability in dynamical systems}\label{sec:intro_predict_dyn}
% Error growth, flow dependence, trajectory vs statistical predictability (short).

\subsection{Sources of uncertainty}\label{sec:intro_unc_sources}
% Initial conditions vs model error; forecast vs analysis uncertainty (short).
% (Optional: table taxonomy of uncertainty sources.)

\subsection{Stochastic parameterizations}\label{sec:intro_stoch}
% Why stochastic physics exists; documented circulation/regime impacts (high-level).

\section{Regimes, persistence, transitions}\label{sec:intro_regimes}

\subsection{Weather regimes as recurrent states}\label{sec:intro_regimes_concept}
% Regimes as persistent flow patterns; positioning blocking as regime(-like).

\subsection{Metastability and transitions}\label{sec:intro_metastability}
% Residence times, transition pathways; why low-order models help.

\subsection{Identifying regimes in practice}\label{sec:intro_regime_id}
% Clustering, HMMs, operator/Markov views; what is used later in the thesis.

\section{Perturbations and persistence}\label{sec:intro_perturb}

\subsection{How perturbations affect lifetimes}\label{sec:intro_perturb_lifetimes}
% Noise/perturbations and escape rates; possible non-monotone effects.
% (Optional: conceptual figure: lifetime vs perturbation amplitude.)

\subsection{Implications for blocking}\label{sec:intro_perturb_blocking}
% Translate metastability to blocking persistence; link to stochastic physics as perturbations.
% Mention Paper I in one sentence if desired.

\section{Uncertainty in the analysis}\label{sec:intro_assim}

\subsection{Data assimilation and analysis uncertainty}\label{sec:intro_assim_basics}
% What an analysis is; why uncertainty is flow-dependent; ensemble viewpoint.

\subsection{Regime-conditioned uncertainty}\label{sec:intro_assim_regime}
% Why regimes provide a coordinate system for uncertainty; hypothesis for blocking.
% Mention Paper II in one sentence if desired.

\subsection{Links to model physics}\label{sec:intro_assim_physics}
% Connecting uncertainty hotspots to model error, stochastic schemes, moist processes.

\section{Moist pathways and a Lagrangian view}\label{sec:intro_moist}

\subsection{Diabatic effects in blocking}\label{sec:intro_diabatic}
% PV modification, ridge maintenance; brief physical mechanism layer.

\subsection{Warm conveyor belts}\label{sec:intro_wcb}
% WCB definition, relevance for blocking and predictability.

\subsection{Lagrangian coherence}\label{sec:intro_coherence}
% Why Lagrangian; coherence/coherent sets for airstreams (conceptual).
% Mention Paper III in one sentence if desired.

\section{Aims and thesis roadmap}\label{sec:intro_roadmap}

\subsection{Research questions}\label{sec:intro_questions}
% 2--4 bullet points.

\subsection{Contributions}\label{sec:intro_contrib}
% 3--6 sentences mapping to Paper I/II/III (short).

\subsection{Thesis structure}\label{sec:intro_structure}
% Chapter-by-chapter overview; use \cref for references to chapters.

% Example (fill in your actual chapter labels):
% \Cref{ch:paper1} ... \Cref{ch:paper2} ... \Cref{ch:paper3} ... \Cref{ch:conclusion} ...
